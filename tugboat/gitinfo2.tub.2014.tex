%!TEX TS-program = latexmkx
%
% --------------------------------------------------------------
\begin{filecontents}{gitinfo2-tub.bib}
@String{ack-bnb = "Barbara N. Beeton,
                   American Mathematical Society,
                   P.O. Box 6248,
                   Providence, RI 02940,
                   USA,
                   Tel: +1 401 455 4014,
                   e-mail: \path|bnb@math.ams.org|"}

@String{ack-nhfb = "Nelson H. F. Beebe,
                    University of Utah,
                    Department of Mathematics, 110 LCB,
                    155 S 1400 E RM 233,
                    Salt Lake City, UT 84112-0090, USA,
                    Tel: +1 801 581 5254,
                    FAX: +1 801 581 4148,
                    e-mail: \path|beebe@math.utah.edu|,
                            \path|beebe@acm.org|,
                            \path|beebe@computer.org| (Internet),
                    URL: \path|http://www.math.utah.edu/~beebe/|"}

@String{j-TUGboat               = "TUGboat"}

@Misc{Latex3:2014:v5471,
  author =       "{The \LaTeX3{} Project}",
  title =        "The l3build package: Checking and building packages",
  month =        nov,
  year =         "2014",
  howpublished = {Available from {\CTAN}, {\url{macros/latex/contrib/l3build/l3build.pdf}}},
  bibdate =      "Sat Dec 12 20:40:00 2014",
}

@Article{Mittelbach:TB35-3-287,
  author =       "Frank Mittelbach and Will Robertson and {The \LaTeX3} Team",
  title =        "{{\tt l3build} \Dash A modern Lua test suite for {\TeX} programming}",
  journal =      j-TUGboat,
  volume =       "35",
  number =       "3",
  pages =        "287--293",
  year =         "2014",
  ISSN =         "0896-3207",
  bibdate =      "Thu Dec 04 17:03:30 GMT 2014",
  acknowledgement = ack-bnb # " and " # ack-nhfb,
}
\end{filecontents}
% --------------------------------------------------------------
\documentclass[]{ltugboat}
% --------------------------------------------------------------
\usepackage{microtype}
\usepackage{url}
\usepackage{cite}
% --------------------------------------------------------------
% Set up a `welter of LaTeX braces' for ``Names of things''
% - - - - - - - - - - - - - - - - - - - - - - - - - - - - - - -
\DeclareUrlCommand\urlsame{\urlstyle{same}}
\newcommand{\afile}[1]{\url{#1}}   % file
\newcommand{\apath}[1]{\url{#1}}   % path
\newcommand{\afext}[1]{\url{#1}}   % file extension
\newcommand{\aprog}[1]{\textsf{\protect\urlsame{#1}}} % program or package
\newcommand{\aluav}[1]{\textit{\textsf{\protect\urlsame{#1}}}} % Lua variable
\newcommand{\lbuild}{\aprog{l3build}}
% --------------------------------------------------------------
\bibliographystyle{plain}
\SetBibJustification{\raggedright}
% --------------------------------------------------------------
\title{\lbuild\ \Dash The Beginner's Tale}
\author{Brent Longborough}
\address{Hightor, Oaks Court\\
  Abersychan, \textsc{np{\footnotesize 4 7}uz}\\
  Wales}
\netaddress{brent@longborough.org}
% --------------------------------------------------------------
\begin{document}
\maketitle
\begin{longabstract}
I stumbled across \lbuild\ by accident,
when I wasn't even dreaming of changing my \CTAN\ workflow.
This is the tale of the first successes of my still-ongoing journey.
\end{longabstract}

% --------------------------------------------------------------
\section{How it all started}
\label{sec:how-it-all}

I maintain a small package on \CTAN,
and for some time had been on the lookout
for tools to help in maintaining it.
My workflow had evolved as far as a \afile{Makefile}
which can handle some special requirements
for the \aprog{git} \acro{DVCS},
typeset the manual,
ensure the file mode bits are appropriately set,
and use \aprog{ctanify} to roll everything up
into a compressed archive for uploading.
I was quite happy with this arrangement,
and wasn't even thinking about changing it.

Then, by coincidence,
two things happened in a single week.
On Tuesday, my \TUB\ arrived,
and I read Frank Mittelbach's
introduction to \lbuild{} \cite{Mittelbach:TB35-3-287}.
Then, on Saturday,
at the \acro{UK-}\tug\ speaker meeting,
Joseph Wright presented \lbuild{}
and described its use by the \LaTeX3 team.
And at that point it occurred to me that this
could be the tool I was looking for,
\emph{and} I could add some formal regression tests.

% --------------------------------------------------------------
\section{The journey so far}
\label{sec:journey-so-far}

As I work on Windows,
and use a full, up-to-date \TeX\ Live
distribution,
\lbuild\ was already in place,
so there was no need to install it.
I read the manual\cite{Latex3:2014:v5471} \emph{(twice)},
then fired up an editor with an empty \afile{build.lua},
and wrote this skeleton:

\begin{verbatim}
  #!/usr/bin/env texlua
  -- Build script for gitinfo2
  module = "gitinfo2"
  -- Need more here?
  kpse.set_program_name("kpsewhich")
  dofile(kpse.lookup("l3build.lua"))
\end{verbatim}

% --------------------------------------------------------------
\subsection{The first step}
\label{sec:getting-started}

`Start simple and add complexity later'.
I started by trying to make an archive with
just the two \afext{.sty} files.
After a couple of false starts,
my first good attempt had these lines:

\begin{verbatim}
  installfiles = {"*.sty"}
  packtdszip   = true
\end{verbatim}

That produced an `interesting' archive.
Only the \apath{doc/} subtree had any files;
\afile{gitinfo2.pdf} had been copied (usefully, but unexpectedly),
along with \afile{post-xxx-sample.txt} with its extension removed.
No \afext{.sty} files were anywhere to be seen \Dash
the \apath{tex/} subtree was empty.

% --------------------------------------------------------------
\subsection{The first mile}
\label{sec:understanding-better}

Back to the manual, where I finally understood that
\verb!installfiles = {"*.sty"}!
was the default.
I had left out \aluav{sourcefiles},
since I had (mistakenly) associated that with
\afext{.ins} and \afext{.dtx} files
\Dash things I don't use.
I also discovered that \afext{.txt} files were included
in the processing of \aluav{readmefiles} by default.

So I had another go:

\begin{verbatim}
  installfiles = {"*.sty"}
  sourcefiles  = {"*.sty"}
  textfiles    = {"post-xxx-sample.txt"}
  readmefiles  = {"README"}
  packtdszip   = true
\end{verbatim}

Now things were looking good.
After a pause to find and fix a small bug in
\aluav{readmefiles} processing,
the build script made me an archive including all of that,
together with \afile{gitinfo2.pdf}.
But I was still uneasy.
The duplication of references to \afext{.sty} files
made me feel uncomfortable,
and the archive \acro{TDS} file included
a \apath{source/} subtree with extra copies of them.

Now, seized by an iron determination to bend the system to my will,
I hacked together a really ugly patch,
and sent it off to a friend on the \LaTeX3 team,
who politely rejected it (quite rightly),
and provided exra function to solve \emph{the} problem,
rather than just \emph{my} problem.
The fix also came with recommendations
on fixing my build configuration to match.

% --------------------------------------------------------------
\subsection{Halfway house}
\label{sec:halfway-house}

With the fix installed, a production-ready
build configuration was now in sight.
All I needed was to bring in an additional text file,
add a demo \afext{.tex} file,
and ensure that the source for the manual would be
included in the archive and typeset at build time.
The core declarations of my definitive \afile{build.lua} are:

\begin{verbatim}
  sourcefiles  = {"*.sty"}
  typesetfiles = {"gitinfo2.tex"}
  textfiles    = {"gitPseudoHeadInfo.gin",
                  "post-xxx-sample.txt"}
  demofiles    = {"gitinfotest.tex"}
  readmefiles  = {"README"}
  packtdszip   = true
  typesetexe   = "xelatex"
  typesetopts  = "-interaction=batchmode"
\end{verbatim}

At this point,
I had a complete replacement for
my \CTAN\ workflow
(apart from minor details specific to
the package concerned),
so I took the day off.

% --------------------------------------------------------------
\subsection{A testing path}
\label{sec:testing-path}

With my \CTAN\ packaging in order,
I now turned to my \emph{principal} reason for
moving to \lbuild \Dash regression testing.
Building my specific test cases is still in the early stages,
and none of them is sufficiently interesting (yet)
to include here in detail.
So far, however, the experience is encouraging \Dash
test cases (once designed) are very easy to build and register.

The first step in preparing tests is to create a subdirectory,
\apath{testfiles/}, in your package directory.
You then create one or more individual test files,
which are \TeX\ or \LaTeX\ files
(specially crafted as described in the \lbuild\ manual),
with the extension \afext{.lvt}.

\lbuild\ regression testing is based on the mechanical
comparison of \TeX\ log files, and
takes place in two parts.
To start, it creates reference logs by typesetting
the test files, one at a time,
and transforming the output log files into a canonical format.
Then, when regression tests are needed,
the tests are typeset again, and the resulting logs canonicalised
and compared with the reference.
If a new logfile does not match its reference log,
then that particular set of tests is considered to have failed,
but all tests are run before any further processing
(such as formatting the manual or building the archive)
is abandoned.

One challenge that I came across fairly quickly is related to
differences in behaviour between \aprog{pdftex} and \aprog{xetex} on the one hand, and \aprog{luatex} on the other.
This example test file
(an \acro{MWE}, with no direct relation to my package)
produced great excitement, differing log output,
and a possible Lua\TeX\ bug:

\begin{verbatim}
  \input regression-test.tex\relax
  \documentclass{article}
  \begin{document}
  \START
  \AUTHOR{Brent Longborough}
  \CLASS[]{none,l3build test}
  \TEST{LuaTeX discrepancy?}{
      \setbox0=\hbox{\textbullet}
      \showbox0
      }
  \end{document}
  \END
\end{verbatim}

To help with the resolution of such differences,
\lbuild\ allows for the creation of individual reference logs,
specific to given engines,
for those cases where the logs are legitimately different.

% --------------------------------------------------------------
\section{Conclusion \Dash\ The journey continues}
\label{sec:journey-continues}

All of this has taken place,
on and off,  
over about six days, during which
I have learned a lot \Dash probably in the wrong order, 
as I started with packaging and ended with test cases,
rather than the other way round. 
Now that I have successfully migrated my workflow to \lbuild,
I have to buckle down to writing test cases,
so this part of \textit{`The Beginner's Tale'} must end here.

As a fairly unsophisticated end user, 
I feel that \lbuild\ provides a useful, 
simple, and easily configurable tool 
for testing and release workflow.
If you maintain a \CTAN\ package, 
I encourage you to try it.

\subsection*{Acknowledgements}
\label{sec:acknowledgements}

I'd like to thank Joseph Wright and Frank Mittelbach
for their suggestions, encouragement, and code.
And, of course, all those other giants of \TeX,
on whose shoulders many of us stand.
% --------------------------------------------------------------
\bibliography{gitinfo2-tub}

\makesignature
\end{document}
