%!TEX TS-program = latexmkx
% 
\documentclass[]{ltugboat}
% --------------------------------------------------------------
\usepackage{microtype}
\usepackage{url}
\usepackage{cite}
\DeclareUrlCommand\afile{\urlstyle{tt}}
\DeclareUrlCommand\adir{\urlstyle{tt}}
\DeclareUrlCommand\afext{\urlstyle{tt}}
\DeclareUrlCommand\aprog{\urlstyle{tt}}
\DeclareUrlCommand\aluav{\urlstyle{sf}}
\newcommand{\lbuild}{{\sf l3build}}
\bibliographystyle{plain}
\SetBibJustification{\raggedright}
% --------------------------------------------------------------
\title{\lbuild\ \Dash The Beginner's Tale}
\author{Brent Longborough}
\address{Abersychan, South Wales}
\netaddress{brent@longborough.org}
% --------------------------------------------------------------
\begin{document}
\maketitle
\begin{longabstract}
I stumbled across \lbuild\ by accident,
when I wasn't even dreaming of changing my \CTAN\ workflow.
This is the tale of the first part of my still-ongoing journey.
\end{longabstract}

% --------------------------------------------------------------
\section{How it all started}
\label{sec:how-it-all}

I maintain a small package on \CTAN, 
and for some time had been on the lookout 
for tools to help in maintaining it.
My workflow had evolved as far as a \afile{Makefile}
which can handle some special requirements 
for the \aprog{git} \acro{DVCS},
typeset the manual, 
ensure the file mode bits are appropriately set, 
and use \aprog{ctanify} to roll everything up 
into a compressed archive for uploading.
I was quite happy with this arrangement, 
and wasn't even thinking about changing it.

Then, by coincidence, 
two things happened in a single week.
On Tuesday, my \TUB\ arrived,
and I read Frank Mittelbach's 
introduction to \lbuild{} \cite{Mittelbach:TB35-3-287}.
Then, on Saturday, 
at the \acro{UK-}\tug\ speaker meeting,
Joseph Wright presented \lbuild{} 
and described its use by the \LaTeX3 team. FIXME 
And at that point it occurred to me that this 
could be the tool I was looking for, 
\emph{and} I could add some formal regression tests.

% --------------------------------------------------------------
\section{The journey so far}
\label{sec:journey-so-far}

As I work on Windows, 
and use a full, up-to-date \TeX\ Live FIXME
distribution, 
\lbuild\ was already in place, 
so there was no need to install it.
I read the manual\cite{Latex3:2014:v5471} \emph{(twice)}, 
then fired up an editor with an empty \afile{build.lua.} 

% --------------------------------------------------------------
\subsection{The first step}
\label{sec:getting-started}

On the principle of `start simple and add complexity later',
I started by trying to make an archive with 
just the two \afext{.sty} files.
After a couple of false starts,
my first attempt looked like this:

\begin{verbatim}
#!/usr/bin/env texlua
-- Build script for gitinfo2
module = "gitinfo2"
  installfiles = {"*.sty"} 
  packtdszip   = true
kpse.set_program_name("kpsewhich")
dofile(kpse.lookup("l3build.lua"))
\end{verbatim}

That produced an `interesting' archive.
Only the \adir{doc/} subtree had any files;
\afile{gitinfo2.pdf} had been copied (usefully, but unexpectedly), 
along with \afile{post-xxx-sample.txt} with its extension removed. 
No \afext{.sty} files were anywhere to be seen \Dash
the \adir{tex/} subtree was empty.

% --------------------------------------------------------------
\subsection{The first mile}
\label{sec:understanding-better}

Back to the manual, where I finally understood that
\verb!installfiles = {"*.sty"}!
was the default.
I had left out \aluav{sourcefiles}, 
since I had (mistakenly) associated that with 
\afext{.ins} and \afext{.dtx} files 
\Dash things I don't use.

I also discovered that \afext{.txt} files were included
in the processing of \aluav{readmefiles} by default.

So I had another go:

\begin{verbatim}
#!/usr/bin/env texlua
-- Build script for gitinfo2
module = "gitinfo2"
  installfiles = {"*.sty"} 
  sourcefiles  = {"*.sty"}
  textfiles    = {"post-xxx-sample.txt"}
  readmefiles  = {"README"}
  packtdszip   = true
kpse.set_program_name("kpsewhich")
dofile(kpse.lookup("l3build.lua"))
\end{verbatim}

Now things were looking good.
After a pause to find and fix a small bug in
\aluav{readmefiles} processing, 
the build script made me an archive including all of that,
together with \afile{gitinfo2.pdf.} 
   


% --------------------------------------------------------------
\subsection{Halfway house}
\label{sec:halfway-house}

% --------------------------------------------------------------
\subsection{A testing path}
\label{sec:testing-path}

% --------------------------------------------------------------
\section{Conclusion \Dash\ The journey continues}
\label{sec:journey-continues}

% --------------------------------------------------------------
% --------------------------------------------------------------
% --------------------------------------------------------------
% --------------------------------------------------------------
% --------------------------------------------------------------
\bibliography{gitinfo2-tub}

\makesignature
\end{document}
